\documentclass[border=30pt, multi, tikz]{standalone} 
\usepackage{import}
\usepackage{amsmath}
\usepackage{graphicx} % Para incluir imágenes
\subimport{layers/}{init}
\usetikzlibrary{positioning}
\usetikzlibrary{3d} % for including external image 

\def\SEColor{rgb:blue,5;green,2.5;white,5}
\def\ConvColor{rgb:yellow,5;red,2.5;white,5}
\def\PoolColor{rgb:red,1;black,0.3}
\def\FcColor{rgb:green,5;blue,5;white,5}
\def\OutColor{rgb:blue,5;green,7;black,2}
\def\WindowColor{rgb:blue,4;yellow,1;black,0.5}

\newcommand{\copymidarrow}{\tikz \draw[-Stealth, line width=0.8mm, draw={rgb:blue,4;red,1;green,1;black,3}] (-0.3,0) -- ++(0.3,0);}

\begin{document}
\begin{tikzpicture}
\tikzstyle{connection}=[ultra thick, draw=\edgecolor, opacity=0.5]

\node[canvas is zy plane at x=0] (temp) at (-1.7,0.4,2.2) {\includegraphics[width=3.857cm,height=2.571cm]{left_eye.jpg}};
\node[canvas is zy plane at x=0] (temp) at (-1.7,0.4,-2) {\includegraphics[width=3.857cm,height=2.571cm]{right_eye.jpg}};

\pic[shift={(0,0.2,0)}] at (0,0,0) 
    {Box={
        name=window,
        caption=\Large \textbf{$\times 2$ CONV EYES} \\ \large {(Shared weights)},
        fill=\WindowColor,  % Color verdoso
        height=25,
        width=35,
        depth=34.8,
        opacity=0.05}
    };

% Conector entre la imagen el bloque de capas VGG16
\draw [connection]  (-1.4,0.4,2.6)    -- node {\midarrow} (0,0.4,2.6);
\draw [connection]  (-1.4,0.4,-1)    -- node {\midarrow} (0,0.4,-1);

\pic[shift={(0,0,0)}] at (0,0,0) 
    {Box={
        name=vgg1,
        fill=\ConvColor,
        height=12.8,
        width=2,
        depth=19.2        }
    };
\pic[shift={(0,0,0)}] at (vgg1-east) 
    {Box={
        name=vgg2,
        fill=\ConvColor,
        height=12.8,
        width=2,
        depth=19.2
        }
    };
\pic[shift={(0,0,0)}] at (vgg2-east) 
    {Box={
        name=vgg3,
        fill=\PoolColor,
        height=6.4,
        width=1,
        depth=9.6,
        opacity=0.5
        }
    };
\pic[shift={(0,0,0)}] at (vgg3-east) 
    {Box={
        name=vgg4,
        fill=\ConvColor,
        height=6.4,
        width=4,
        depth=9.6
        }
    };
\pic[shift={(0,0,0)}] at (vgg4-east) 
    {Box={
        name=vgg5,
        fill=\ConvColor,
        height=6.4,
        width=4,
        depth=9.6
        }
    };

% Bloque de convoluciones adicionales (sin conexiones y sin nombres de capas)
\pic[shift={(1.5,0,0)}] at (vgg5-east) 
    {Box={
        name=conv_add1,
        fill=\ConvColor,
        height=6.4,
        width=2,
        depth=9.6
        }
    };

% Conector entre el bloque VGG16 y el siguiente bloque de convoluciones
\draw [connection]  (vgg5-east)    -- node {\midarrow} (conv_add1-west);

\pic[shift={(0,0,0)}] at (conv_add1-east) 
    {Box={
        name=conv_add2,
        fill=\ConvColor,
        height=5.6,
        width=2,
        depth=8.8
        }
    };
\pic[shift={(0,0,0)}] at (conv_add2-east) 
    {Box={
        name=conv_add3,
        fill=\ConvColor,
        height=4.4,
        width=2,
        depth=7.6
        }
    };
\pic[shift={(0,0,0)}] at (conv_add3-east) 
    {Box={
        name=conv_add4,
        fill=\ConvColor,
        height=2.4,
        width=4,
        depth=5.6
        }
    };
\pic[shift={(0,0,0)}] at (conv_add4-east) 
    {Box={
        name=conv_add5,
        fill=\ConvColor,
        height=0.8,
        width=4,
        depth=1.2
        }
    };

% Concatenación de los dos bloques en una esfera (Ball)
\pic[shift={(1.7,-0.3,0)}] at (window-east) {Ball={name=concat, fill=\OutColor, opacity=0.6, radius=1.5, logo=$+$}};

%Conexiones a la bola de concatenación
\draw [connection] (8.3,1,1.5) -- node {\midarrow}  ++(1.5,0,0) -- (concat-ne);
\draw [connection] (8.3,1,4.5) -- node {\midarrow}  ++(1.5,0,0) -- (concat-sw);

%Bloque de capas SE + Convolucionales
\pic[shift={(1.5,0,0)}] at (concat-east) 
    {Box={
        name=SE1,
        fill=\SEColor,
        height=0.8,
        width=4,
        depth=1.2
        }
    };
% Conector entre el bloque VGG16 y el siguiente bloque de convoluciones
\draw [connection]  (concat-ee) -- node {\midarrow} (SE1-west);

\pic[shift={(0,0,0)}] at (SE1-east) 
    {Box={
        name=conv_add6,
        fill=\ConvColor,
        height=0.8,
        width=4,
        depth=1.2
        }
    };
\pic[shift={(0,0,0)}] at (conv_add6-east) 
    {Box={
        name=SE2,
        fill=\SEColor,
        height=0.8,
        width=4,
        depth=1.2
        }
    };
\pic[shift={(0,0,0)}] at (SE2-east) 
    {Box={
        name=conv_add7,
        fill=\ConvColor,
        height=0.8,
        width=2,
        depth=1.2
        }
    };
\pic[shift={(0,0,0)}] at (conv_add7-east) 
    {Box={
        name=SE3,
        fill=\SEColor,
        height=0.8,
        width=2,
        depth=1.2
        }
    };

\pic[shift={(1.5,0,0)}] at (SE3-east)  
    {Box={
        name=fc1,
        caption=FC1,
        xlabel={{"1", }},
        zlabel=\hfill 512,
        fill=\FcColor,
        height=0.65,
        width=0.65,
        depth=18
        }
    };
% Conector entre el bloque SE+CONV y el siguiente bloque de fc1
\draw [connection]  (SE3-east) -- node {\midarrow} (fc1-west);


% Nombre VGG16 
\node[anchor=south] at (0.4,-3.2) {  \begin{tabular}{c}
    \textbf{VGG16} \\
    (first 9 layers)
  \end{tabular}
};
% Nombre del bloque de convoluciones adicionales
\node[anchor=south] at (4.2,-2.6) {
  \begin{tabular}{c}
    \textbf{SEQ CONV} \\
    (dilation convs)
  \end{tabular}
};
% Nombre concatenacion
\node[anchor=south] at (8,-2) {\begin{tabular}{c} \textbf{CONCAT} \end{tabular}
};
% Nombre bloque SE + Conv
\node[anchor=south] at (11.8,-2) {
  \begin{tabular}{c}
    \textbf{SE LAYERS +} \\
    \textbf{CONVS}
  \end{tabular}
};

% Tamaño final de bloques
\node[anchor=south, rotate=45] at (-2.4,1) {96x64x3};
\node[anchor=south, rotate=45] at (-0.8,2.6) {96x64x3};
\node[anchor=south] at (1.9,2.1) {$96\times96\times64$};
\node[anchor=south] at (3,1.1) {$48\times32\times128$};
\node[anchor=south] at (6.7,0.15) {$6\times4\times128$};
\node[anchor=south] at (8.8,0.7) {$6\times4\times256$};
\node[anchor=south] at (13.2,0.15) {$6\times4\times128$};

\end{tikzpicture}
\end{document}
